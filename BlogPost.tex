% Options for packages loaded elsewhere
% Options for packages loaded elsewhere
\PassOptionsToPackage{unicode}{hyperref}
\PassOptionsToPackage{hyphens}{url}
\PassOptionsToPackage{dvipsnames,svgnames,x11names}{xcolor}
%
\documentclass[
  letterpaper,
  DIV=11,
  numbers=noendperiod]{scrartcl}
\usepackage{xcolor}
\usepackage{amsmath,amssymb}
\setcounter{secnumdepth}{-\maxdimen} % remove section numbering
\usepackage{iftex}
\ifPDFTeX
  \usepackage[T1]{fontenc}
  \usepackage[utf8]{inputenc}
  \usepackage{textcomp} % provide euro and other symbols
\else % if luatex or xetex
  \usepackage{unicode-math} % this also loads fontspec
  \defaultfontfeatures{Scale=MatchLowercase}
  \defaultfontfeatures[\rmfamily]{Ligatures=TeX,Scale=1}
\fi
\usepackage{lmodern}
\ifPDFTeX\else
  % xetex/luatex font selection
\fi
% Use upquote if available, for straight quotes in verbatim environments
\IfFileExists{upquote.sty}{\usepackage{upquote}}{}
\IfFileExists{microtype.sty}{% use microtype if available
  \usepackage[]{microtype}
  \UseMicrotypeSet[protrusion]{basicmath} % disable protrusion for tt fonts
}{}
\makeatletter
\@ifundefined{KOMAClassName}{% if non-KOMA class
  \IfFileExists{parskip.sty}{%
    \usepackage{parskip}
  }{% else
    \setlength{\parindent}{0pt}
    \setlength{\parskip}{6pt plus 2pt minus 1pt}}
}{% if KOMA class
  \KOMAoptions{parskip=half}}
\makeatother
% Make \paragraph and \subparagraph free-standing
\makeatletter
\ifx\paragraph\undefined\else
  \let\oldparagraph\paragraph
  \renewcommand{\paragraph}{
    \@ifstar
      \xxxParagraphStar
      \xxxParagraphNoStar
  }
  \newcommand{\xxxParagraphStar}[1]{\oldparagraph*{#1}\mbox{}}
  \newcommand{\xxxParagraphNoStar}[1]{\oldparagraph{#1}\mbox{}}
\fi
\ifx\subparagraph\undefined\else
  \let\oldsubparagraph\subparagraph
  \renewcommand{\subparagraph}{
    \@ifstar
      \xxxSubParagraphStar
      \xxxSubParagraphNoStar
  }
  \newcommand{\xxxSubParagraphStar}[1]{\oldsubparagraph*{#1}\mbox{}}
  \newcommand{\xxxSubParagraphNoStar}[1]{\oldsubparagraph{#1}\mbox{}}
\fi
\makeatother


\usepackage{longtable,booktabs,array}
\usepackage{calc} % for calculating minipage widths
% Correct order of tables after \paragraph or \subparagraph
\usepackage{etoolbox}
\makeatletter
\patchcmd\longtable{\par}{\if@noskipsec\mbox{}\fi\par}{}{}
\makeatother
% Allow footnotes in longtable head/foot
\IfFileExists{footnotehyper.sty}{\usepackage{footnotehyper}}{\usepackage{footnote}}
\makesavenoteenv{longtable}
\usepackage{graphicx}
\makeatletter
\newsavebox\pandoc@box
\newcommand*\pandocbounded[1]{% scales image to fit in text height/width
  \sbox\pandoc@box{#1}%
  \Gscale@div\@tempa{\textheight}{\dimexpr\ht\pandoc@box+\dp\pandoc@box\relax}%
  \Gscale@div\@tempb{\linewidth}{\wd\pandoc@box}%
  \ifdim\@tempb\p@<\@tempa\p@\let\@tempa\@tempb\fi% select the smaller of both
  \ifdim\@tempa\p@<\p@\scalebox{\@tempa}{\usebox\pandoc@box}%
  \else\usebox{\pandoc@box}%
  \fi%
}
% Set default figure placement to htbp
\def\fps@figure{htbp}
\makeatother





\setlength{\emergencystretch}{3em} % prevent overfull lines

\providecommand{\tightlist}{%
  \setlength{\itemsep}{0pt}\setlength{\parskip}{0pt}}



 


\KOMAoption{captions}{tableheading}
\makeatletter
\@ifpackageloaded{caption}{}{\usepackage{caption}}
\AtBeginDocument{%
\ifdefined\contentsname
  \renewcommand*\contentsname{Table of contents}
\else
  \newcommand\contentsname{Table of contents}
\fi
\ifdefined\listfigurename
  \renewcommand*\listfigurename{List of Figures}
\else
  \newcommand\listfigurename{List of Figures}
\fi
\ifdefined\listtablename
  \renewcommand*\listtablename{List of Tables}
\else
  \newcommand\listtablename{List of Tables}
\fi
\ifdefined\figurename
  \renewcommand*\figurename{Figure}
\else
  \newcommand\figurename{Figure}
\fi
\ifdefined\tablename
  \renewcommand*\tablename{Table}
\else
  \newcommand\tablename{Table}
\fi
}
\@ifpackageloaded{float}{}{\usepackage{float}}
\floatstyle{ruled}
\@ifundefined{c@chapter}{\newfloat{codelisting}{h}{lop}}{\newfloat{codelisting}{h}{lop}[chapter]}
\floatname{codelisting}{Listing}
\newcommand*\listoflistings{\listof{codelisting}{List of Listings}}
\makeatother
\makeatletter
\makeatother
\makeatletter
\@ifpackageloaded{caption}{}{\usepackage{caption}}
\@ifpackageloaded{subcaption}{}{\usepackage{subcaption}}
\makeatother
\usepackage{bookmark}
\IfFileExists{xurl.sty}{\usepackage{xurl}}{} % add URL line breaks if available
\urlstyle{same}
\hypersetup{
  pdftitle={BlogPost},
  pdfauthor={Corbin Brinkerhoff, Matthew Wilson},
  colorlinks=true,
  linkcolor={blue},
  filecolor={Maroon},
  citecolor={Blue},
  urlcolor={Blue},
  pdfcreator={LaTeX via pandoc}}


\title{BlogPost}
\author{Corbin Brinkerhoff, Matthew Wilson}
\date{}
\begin{document}
\maketitle


\section{What Should You Really Expect to Earn Five Years After
College?}\label{what-should-you-really-expect-to-earn-five-years-after-college}

\begin{center}\rule{0.5\linewidth}{0.5pt}\end{center}

\subsection{1. Introduction}\label{introduction}

Choosing a college major or stressing over grades can feel like
decisions that will shape the rest of your life, and many students
wonder what these choices actually mean for their future income. To help
answer these questions, we analyzed real data from the American
Community Survey to understand how college graduates are doing five
years after earning their degree. In this article, we break down how
factors like GPA, major, and gender relate to salary in the early career
years without the confusing technical language. Whether you're a current
student, a recent graduate, or someone advising the next generation,
this guide is designed to give you a clear, data‑driven look at what
influences earnings and what matters less than you might think.

\begin{center}\rule{0.5\linewidth}{0.5pt}\end{center}

\subsection{2. Impact of GPA}\label{impact-of-gpa}

When looking at how grades relate to future earnings, the data shows a
real, though not overwhelming, connection. On average, each one‑point
increase in GPA is linked to earning about \$5,400 more per year, and
the true effect is very likely somewhere between \$4,700 and \$6,100. So
yes, GPA does matter, but not in the ``your life depends on perfect
grades'' way that many students fear. A strong GPA can give you a
helpful boost, but it's far from the only factor shaping your salary
after graduation. Interestingly, the importance of GPA also shifts
depending on what you study. While differences across majors aren't
huge, some fields reward high academic performance more consistently
than others. In these majors, employers may view GPA as a stronger
signal of technical skill or work ethic. In other areas, experience,
internships, or portfolios might speak louder than a transcript.
Overall, GPA matters but it shouldn't be a source of extreme stress.
It's one piece of a much bigger career puzzle.

\begin{center}\rule{0.5\linewidth}{0.5pt}\end{center}

\subsection{3. Salaries Across Majors}\label{salaries-across-majors}

\begin{itemize}
\item
  \textbf{Differences in salary by major:}\\
  \{\{State whether your dataset shows significant salary differences
  among majors.\}\}
\item
  \textbf{High‑Earning Majors:}\\
  \{\{List or describe which majors appear to earn the highest salaries
  based on your data.\}\}
\end{itemize}

\pandocbounded{\includegraphics[keepaspectratio]{BlogPost_files/figure-pdf/unnamed-chunk-1-1.pdf}}

\begin{verbatim}
There is a difference in salary for different majors. We found that the average salary 5 years after graduating is highest for **Engineering, Computers and Mathematics, and Physical Sciences.**
\end{verbatim}

\begin{center}\rule{0.5\linewidth}{0.5pt}\end{center}

\subsection{4. Salaries for Women and
Men}\label{salaries-for-women-and-men}

When comparing salaries between men and women within the same major
category, the data reveals a persistent difference. Even after
accounting for GPA and field of study, women earn about \$2,700 less per
year than men on average. This gap doesn't look the same in every major,
though some fields show very little difference, while others show
noticeably larger ones. The interaction between gender and major
indicates that in a few specific categories, the gap widens or narrows
depending on the nature of the work or industry norms. For example, in
some major categories, the salary difference swings by an additional
\$1,000--\$2,300, either reducing or increasing the gap. Looking at your
own major category, these patterns would highlight whether your field
tends to reward men and women equally or whether a meaningful difference
appears. While the exact reasons behind these gaps can be complex, the
takeaway is clear: salary differences between men and women do exist,
and they aren't the same across all majors. Understanding these patterns
can help students make informed decisions and advocate more confidently
when entering the job market.

\pandocbounded{\includegraphics[keepaspectratio]{BlogPost_files/figure-pdf/unnamed-chunk-2-1.pdf}}

\begin{center}\rule{0.5\linewidth}{0.5pt}\end{center}

\subsection{5. Other Factors --- The Value of a College
Education}\label{other-factors-the-value-of-a-college-education}

\subsubsection{Problem Background}\label{problem-background}

\begin{itemize}
\item
  \textbf{Predictors of Salary:}\\
  \{\{Describe how well your dataset's existing variables explain or
  predict salary outcomes.\}\}
\item
  \textbf{Missing or Additional Variables:}\\
  \{\{List important factors not included in the dataset that likely
  influence salary (e.g., internships, geographic location, work
  experience).\}\}
\item
  \textbf{Specific Example Demonstrating Prediction Quality:}\\
  \{\{Walk through one specific example that illustrates how well---or
  poorly---your model or variables predict salary.\}\}

  Our model will predict the salary of a person, and on average be off
  by about \$5700. We currently are using a person's Major, GPA, and sex
  to predict their salary. There are other variables that would
  potentially help us to better predict salary, including but not
  limited to times changing industries, continued eduction (i.e.~a
  Master's Program), and geographic locations. All of these are factors
  that partially determine a salary. Another potential determinant is
  how much they liked their major, as a person who likes their work is
  more likely to perform well and get raises. To give an example of our
  predicition compared to the true salaries, we randomly selected a
  person from the dataset, and predicted their salary. We predicted a
  salary of ``\$XXXX'', and they had a salary of XXXX dollars. So we
  were off by XXXX dollars.
\end{itemize}

\begin{center}\rule{0.5\linewidth}{0.5pt}\end{center}

\subsection{6. Conclusions}\label{conclusions}

\begin{itemize}
\item
  \textbf{Key Takeaways:}\\
  \{\{State the main message readers should remember after reading the
  article. Summarize your most important findings in 2--4 sentences.\}\}

  Beats me
\end{itemize}

\begin{center}\rule{0.5\linewidth}{0.5pt}\end{center}




\end{document}
